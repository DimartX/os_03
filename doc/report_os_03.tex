\documentclass[12pt]{article}

\usepackage{fullpage}
\usepackage{multicol,multirow}
\usepackage{tabularx}
\usepackage{ulem}
\usepackage[utf8]{inputenc}
\usepackage[russian]{babel}
\usepackage{minted}

\usepackage{color} %% это для отображения цвета в коде
\usepackage{listings} %% собственно, это и есть пакет listings

\lstset{ %
  language=C,                 % выбор языка для подсветки (здесь это С)
  basicstyle=\small\sffamily, % размер и начертание шрифта для подсветки кода
  numbers=left,               % где поставить нумерацию строк (слева\справа)
  %numberstyle=\tiny,           % размер шрифта для номеров строк
  stepnumber=1,                   % размер шага между двумя номерами строк
  numbersep=5pt,                % как далеко отстоят номера строк от подсвечиваемого кода
  backgroundcolor=\color{white}, % цвет фона подсветки - используем \usepackage{color}
  showspaces=false,            % показывать или нет пробелы специальными отступами
  showstringspaces=false,      % показывать или нет пробелы в строках
  showtabs=false,             % показывать или нет табуляцию в строках
  frame=single,              % рисовать рамку вокруг кода
  tabsize=2,                 % размер табуляции по умолчанию равен 2 пробелам
  captionpos=t,              % позиция заголовка вверху [t] или внизу [b] 
  breaklines=true,           % автоматически переносить строки (да\нет)
  breakatwhitespace=false, % переносить строки только если есть пробел
  escapeinside={\%*}{*)}   % если нужно добавить комментарии в коде
}

\begin{document}
\begin{titlepage}
  \large
  \begin{center} 
    
      Московский Авиационный Интститут \\
      (Национальный Исследовательский Университет) \\
      Факультет информационных технологий и прикладной математики \\
      Кафедра вычислительной математики и программирования \\
      \vfill\vfill
      \textbf{
        { Лабораторная работа №3 по курсу} \\ 
        <<Операционные системы>> \\
        \bigskip
            {Управление потоками и синхронизация } \\
    } \\
  \end{center}
  \vfill

  \begin{flushright}

    Студент:  {Артемьев Дмитрий Иванович}

    Группа: {М8О-206Б-18}

    Вариант: {2}
    
    Преподаватель: {Соколов Андрей Алексеевич}

    Оценка: $\rule{3cm}{0.15mm}$

    Дата: $\rule{3cm}{0.15mm}$
    
    Подпись: $\rule{3cm}{0.15mm}$

  \end{flushright}
  \vfill
  \begin{center}
    Москва, 2019
  \end{center}
  
\end{titlepage}

\subsection*{Условие}

2. Отсортировать массив строк при помощи параллельного алгоритма быстрой сортировки.

\subsection*{Описание программы}

Код программы находится в файле main.c.

\subsection*{Ход выполнения программы}
\begin{enumerate}
\item Запуск программы с параметрами количества потоков depth (2^{depth}), длины сортируемого массива
\item Считывание элементов массива 
\item Запуск параллельной быстрой сортировки
\item \ на определённом уровне рекурсии \ если максимальная глубина рекурсии, на которой можно создавать новые потоки не достигнута, создаются новые потоки, выполняющие сортировку в данном им куске массива. Иначе потоки не создаются и оставшуюся часть работы выполняют существующие потоки.   
\item Вывод результата сортировки.
\item Завершение работы программы.
\end{enumerate}

\subsection*{Недочёты}

Нельзя задавать произвольное количество потоков, создающихся в ходе выполнения.

\subsection*{Выводы}

Я научился работать с потоками в Linux, обеспечивать синхронизацию между ними, решать возникающие состязательные ситуации с помощью mutex, которые, однако, не используются мной в самой лабораторной.
\pagebreak

\vfill

\subsection*{Исходный код}

{\Huge main.cpp}
\inputminted
    {C++}{src/main.cpp}
    \pagebreak    
\end{document}
